\documentclass[final]{beamer}
\usetheme{Berlin}
\usepackage[orientation=portrait,size=a0,scale=1.4,debug]{beamerposter}

\title[FastGates]{Fast gates with trapped ions}

\begin{document}
\begin{frame}{} 

\begin{center}
\begin{columns}[T,onlytextwidth]
  \begin{column}{0.48\textwidth}

    \begin{block}{Why Fast Gates?}
      \begin{itemize}
      \item Define a gate - MS hamiltionian
      \item fig: alkali metal zero spin structure diagram
      \item quadropole transition vs Raman

      \item Long coherence time of ions 
      \item Clock speeds of ion trap qc
      \item issues with scaling (commutivity of quadropole MS hamiltionian bessel) and proposed solution
      \item issue with exciting spectators. Use fancy pulse shaping (Vera paper).
      \item fig: pulse shaping
      \item fig: interaction strength with motion saturates.
      \end{itemize}
    \end{block}


    \begin{block}{Fast Gates with a Lattice}
      \begin{itemize}
      \item MS Hamiltionian with lattice removes this saturation.
      \item choosing phase diff of 0 and sitting at antinode to get maximal sb coupling
      \item fig: Control of phase visible to ions
      \item fig: MS gate fidelitly with gate time
      \end{itemize}
    \end{block}

  \end{column}


  \begin{column}{0.48\textwidth}

    \begin{block}{Phase Stabilization}
      \begin{itemize}
      \item fig: experimental set up (fig1 cnulled)
      \item using ion feedback
      \item fig: RBM data to show no worse
      \end{itemize}
    \end{block}


    \begin{block}{New Platform}
      \begin{itemize}
      \item whats new: double NA, Ca40, NPL trap (3D heating rates), MuMetal shield, perm magnets.
      \item quadropole used.

        ``In practice, the dominant error source in [quadropole] gates is laser
        frequency noise resonant with the carrier transition,
        uncontrolled light shifts arising from the carrier, and laser
        phase noise at time scales comparable with the gate; all of
        these are exacerbated by the relatively small Lamb-Dicke
        parameter (typically ~0.05), which also sets a practical limit
        to the gate speed because it limits gates to the adiabatic
        regime [Roos 2008].''

      \item fig: solidworks of new experiment
      \item fig: array of single addressing SW
      \end{itemize}
    \end{block}

  \end{column}
\end{columns}
\end{center}

\end{frame}
\end{document}
