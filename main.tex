%Settings {{{
\documentclass[final]{beamer}
\usepackage[orientation=portrait,size=a0,scale=1,debug]{beamerposter}
\usepackage[T1]{fontenc}
\usepackage{lmodern}
\usetheme{gemini}
\usecolortheme{ox}
\usepackage{graphicx}
\usepackage{booktabs}
\usepackage{tikz}
\usepackage{pgfplots}
\pgfplotsset{compat=1.14}
\usepackage{anyfontsize}

%\setsansfont{FoundSteBoo}
%default font setup for use with pdflatex
%\renewcommand{\rmdefault}{phv} %%% Helvetica (very similar to Arial)


% ====================
% Lengths
% ====================

% If you have N columns, choose \sepwidth and \colwidth such that
% (N+1)*\sepwidth + N*\colwidth = \paperwidth
\newlength{\sepwidth}
\newlength{\colwidth}
\setlength{\sepwidth}{0.025\paperwidth}
\setlength{\colwidth}{0.3\paperwidth}

\newcommand{\separatorcolumn}{\begin{column}{\sepwidth}\end{column}}

\footercontent{
  XXXReferences and fundersXXX \hfill
  SAMOP 2023, \hfill
  \href{mailto:donovan.webb@physics.ox.ac.uk}{donovan.webb@physics.ox.ac.uk}}

% ====================
% Logo (optional)
% ====================

% use this to include logos on the left and/or right side of the header:
% \logoright{\includegraphics[height=7cm]{logo1.pdf}}
\logoleft{\fbox{
    \includegraphics[height=7cm]{figs/oxlogo.pdf}}
}

\newcommand{\SubItem}[1]{
    {\setlength\itemindent{15pt} \item[] #1}
}

%}}}

%title {{{

\title[FastGates]{\Huge Next generation platform for implementing fast gates in ion trap quantum computation}
\author{\textbf{D. Webb \and S. Saner \and O. Bazavan \and M. Minder \and C.J. Ballance}}
\institute[]{
Ion Trap Quantum Computing Group,
Department of Physics, University of Oxford}

\begin{document}
\begin{frame}{} 
%}}}

\begin{center}

%Abstract {{{

    \vspace{-1em}
    \begin{block}{}
    \large
    Scalable trapped-ion quantum computation relies on the development of
    high-fidelity fast entangling gates in a many ion
    crystal. Conventional geometric phase gates either suffer from
    scattering errors or off-resonant carrier excitations. A potential
    route to achieve fast entanglement is creating a standing wave which
    can suppress the unwanted carrier coupling. \\

    %% We present the roadmap to our next-generation platform tailored for
    %% fast gates in the ~1us regime where gate speeds become comparable to
    %% the secular trap frequency. The quadrupole transitions between S1/2
    %% and D5/2 levels in Calcium 40 will be driven to perform
    %% Molmer-Sorenson gates with a standing wave rather than a typical
    %% travelling wave. The off-resonant carrier excitation may be strongly
    %% suppressed by placing ions at the nodes of the optical lattice. This
    %% new platform has scope for a multi-ion chain and a corresponding array
    %% of optical lattices which each address a single ion. The lattice array
    %% is created by a set of counter-propagating beams which are tightly
    %% focused by a symmetric setup of high-NA lenses. Control of the optical
    %% phase at the ion site will be achieved by actively stabilising the
    %% counter-propagating beam interferometer and feedbacking on the ion
    %% signal.
    \end{block}
%}}}

\begin{columns}[t]
  \begin{column}{0.49\textwidth}

%Why {{{
    \begin{alertblock}{Non-Adiabatic Mølmer Sørenson Gates}
      \begin{itemize}
      \item Two qubit gates are implemented by coupling spin with shared motion of the ions.
        %XXX Figure of geometric phase gate. XXX
      \item Fast entangling gates enable deeper quantum computational circuit depths for given level of incoherent error.
      \item But going fast excites multiple motional modes (``spectator modes'') which can introduces errors.
        %XXX Figure of multiple motions. XXX
        
      \item Mølmer Sørenson (MS) interaction is common two-qubit gate which requires a
      bichromatic field incident on the ions. Using travelling
      waves gives the Hamiltonian
      \Large$$ \hat{H}_{MS-TW} = \hbar\Omega \hat{S}_{\phi-\pi/2}\cos{(\delta t)} + \hbar\Omega\eta \hat{S}_\phi\cos{(\delta t)}(\hat{a}e^{-i\omega_zt} + \hat{a}^\dagger e^{i\omega_zt})$$\normalsize
      \vspace{0.4em}
      with the first term being the carrier whilst the second is the desired coupling.\vspace{0.8em} 
        %XXX Figure of TW MS . XXX

      \item As these terms do not commute, in the ??interaction picture?? this Hamiltonian
            may be expressed as [Xref Canzz?X]:
      \Large$$ \hat{H}_{MS-TW} = \hbar\eta\Omega(J_0(2\Omega/\delta) + J_2(2\Omega/\delta))\cdot \cos{(\delta t)}\hat{S}_{\phi}(\hat{a}e^{-i\omega_zt} + \hat{a}^\dagger e^{i\omega_zt})$$\normalsize
      \end{itemize}
      \begin{minipage}{0.58\textwidth}
      \begin{itemize}
      \item \textbf{Carrier term:} causes the spin dependent force
        coupling to be modulated by (J0+J2).

      \item \textbf{Spectator excitation:} Amplitude shaped pulses
        [ref vera] effectively remove ``spectator'' error by closing
        phase loops of all excited modes.
        
      \end{itemize}
      \end{minipage}
      \begin{minipage}{0.38\textwidth}
      \begin{figure}
        \includegraphics[width=0.98\textwidth]{./figs/J0J2.pdf}
        %XXX Use theory graph
        %XXX CANZZ aside box
      \end{figure}
      \end{minipage}

      \begin{figure}
        \includegraphics[width=0.9\textwidth]{./figs/loop_closing.png}
      \end{figure}
%}}}

%Lattice {{{
    \heading{Standing Wave Single and MS Gates}
      \begin{itemize}
      \item Using standing wave gives complete control over phase
        visible to ions.
      \item This extra freedom allows fast gates by preventing the
        saturation effect seen in the travelling MS.
      \item Bichromatic standing wave Hamiltonian where ions are
        seperated by $n\lambda/2$:
      \large$$ \hat{H}_{MS-SW} = \hbar 2\Omega \hat{S}_{\phi}\cos{(\delta t)}\sin{(\Delta\phi/2)} + \hbar 2\Omega\eta \hat{S}_\phi\cos{(\delta t)}(\hat{a}e^{-i\omega_zt} + \hat{a}^\dagger e^{i\omega_zt})\cos{(\Delta\phi/2)}$$\normalsize
        %XXX Figure of SW MS (1b) . XXX
      \item Setting $\Delta\phi = 0$ (ions sitting at antinodes) we
        suppress the carrier term and maximise sideband
        coupling.
      \item However standing waves require phase stabilisation to perform
            coherent interactions.
      \end{itemize}
    \end{alertblock}
%}}}

%Phase stabilization {{{
    \begin{alertblock}{Phase Stabilization}
      \begin{figure}
        \includegraphics[width=0.5\textwidth]{./figs/setup+beams_horizontal.pdf}
      \end{figure}

      \begin{itemize}
      \item Passively using enclosure and actively by feedback process to AOM.\\
      \SubItem 1) Fast drifts removed utilising interference of light
            from the two branches on a photodiode (PD).\\

      \SubItem XXX diagram rather than words. 2) As PD lockpoint is ~30 cm away
            from ions, a second feedback loop using the ion as a
            sensor is required. We do this by performing a $\pi/2$-pulse
            using arm 1 ($\pi/2$, b1) followed immediately by a $\pi/2$-pulse
            using arm 2 ($pi/2$, b2). This is equivalent to a zero-delay
            Ramsey sequence, which gives a signal sensitive on the
            difference in phase between the two pulses, hence the
            relative phase be- tween the branches.\\

      \item No detriment using lattice: Standard Randomized
            Benchmarking gave quality of our single qubit rotations per
            gate to be $\epsilon = 0.173(3)\%$ for a travelling wave and
            $\epsilon = 0.144(3)\%$ for the standing wave.\\
      \end{itemize}
    \end{alertblock}
%}}}

%Experimental {{{
  \end{column}
  \begin{column}{0.49\textwidth}
    \begin{alertblock}{Experimental Results}
      Results from Cnulling:
      \begin{figure}
        \includegraphics[width=0.48\textwidth]{./figs/Figure_2_v2.pdf}
        \includegraphics[width=0.48\textwidth]{./figs/two_qubit_gate_figure.pdf}
      \end{figure}
    \end{alertblock}


    \begin{alertblock}{New Platform}
      \begin{minipage}{0.58\textwidth}
      \begin{itemize}
      \item New ion trap experiment in development for exploring fast
        gate regime.
      \item Two high NA lenses allow optical
        access on both faces of the trap to create an array of singly
        addressing standing waves.
      \item 3D segmented trap design from NPL facilitates low heating
        rates whilst enabling ion shuttling and crystal rotations.
      \item Calcium 40
      \item quadropole transition used.
      \item MuMetal shield and permanent magnets.
      \item fig: solidworks of new experiment
      \end{itemize}
      \end{minipage}
      ~~
      \begin{minipage}{0.35\textwidth}
      \includegraphics[width=0.94\textwidth]{./figs/ca_struct_tmp.jpeg}
      \end{minipage}

      \begin{figure}
        \includegraphics[height=0.35\textwidth]{./figs/trap_NA.png}
        \includegraphics[height=0.35\textwidth]{./figs/exp_iso.png}
        \includegraphics[height=0.35\textwidth]{./figs/array_sw.png}
      \end{figure}

    \end{alertblock}
%}}}

  \end{column}
\end{columns}
\end{center}

\end{frame}
\end{document}
